\setcounter{step}{0}
%------------------------------------------
% information doc
\subsection{Palacinky}
\PrepTime{10}
\CookingTime{5}
\CookingTempe{180}
\TypeCooking{Vyprážanie na sucho? dačo}
\NbPerson{1}
\Image{0 0 430 430}{images/florentin} %style 2
%------------------------------------------

\begin{ingredient}
%\vspace{0.5cm}
\begin{main}
	\item 1 vajce
	\item 100g hladkej múky
	\item 100ml mlieko
	\item 1/4 vanilkového cukru, štipka soli
\end{main}
\end{ingredient}%no space with \begin{recipe}
\begin{recipe}



\step{Všetky zmiešame dokopy}
\step{Opekáme na panvici}

\end{recipe}

\begin{notes}

\end{notes}	