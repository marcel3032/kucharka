\setcounter{step}{0}

\subsection{ Florentins au chocolat }

\begin{ingredient}
  
  \def\portions{  }
  \textbf{ {\normalsize Ingrediencie (4 porcie):} }

  \begin{main}
      \item 4 ca.S de sucre en poudre
      \item 1 ca.S de creme liquide
      \item 1 ca.S de miel
      \item 1 grosse noix de beurre
      \item 35 gr d’amandes effilées
      \item 50 gr de chocolat au lait ou chocolat blanc
  \end{main}
  
    \begin{subingredient}{Test subingredient}
        \item 1 ca.c de test1
        \item 1 a 2 ca.S de test2
        \item 3 gouttes de test3
        \item 8 morceaux de test4.
    \end{subingredient}
  
\end{ingredient}
\begin{recipe}
\textbf{ {\normalsize Príprava:} }
\begin{enumerate}

  \item{Préchauffez votre four a 180°C (th.6).}
  \item{Dans une casserole, faites bouillir le sucre en poudre avec la creme liquide, le beurre et le miel.}
  \item{Une fois que le sucre prend une jolie coloration brune, versez les amandes effilées dans la casserole, et remuez bien pour napper l’intégralité des amandes.}
  \item{Pour la cuisson au four vous avez 2 possibilités : Soit vous versez la « pâte » dans le fond de moules en silicone, type moules a muffins ou moules a tartelettes, soit vous étalez bien la « pâte », et rapidement car le caramel durcit vite, sur la plaque de votre four recouverte d’une feuille de papier sulfurisé, et apres la cuisson vous découperez des cercles a l’aide d’un emporte-pieces rond.}
  \item{Dans tous les cas, mettez la « pâte » au four pendant 3 a 5 minutes. A la sortie du four, soit vous découpez tout de suite des ronds a l’aide de l’emporte-pieces, soit vous laissez refroidir les florentins avant de les démouler de vos moules a muffins.}
  \item{Pendant que les florentins refroidissent, faites fondre le chocolat au lait ou blanc soit au bain-marie, soit au micro-ondes a faible puissance, soit dans une petite casserole a feu doux.}
  \item{Trempez ensuite la moitié des florentins dans le chocolat fondu et mettez-les au réfrigérateur pendant une bonne vingtaine de minutes pour que le chocolat prenne bien.}
  \item{Test substep: }
      \begin{enumerate}
          \item{Blabla}
          \item{Blabla}\end{enumerate}

\end{enumerate}
\end{recipe}

\begin{notes}
  
\end{notes}	
\clearpage
